\documentclass[a4paper,12pt]{article} % Hoặc article nếu là sách ngắn
\usepackage[left=10mm, right=10mm, top=30mm, bottom=30mm]{geometry}
\usepackage[english, vietnamese]{babel}
\usepackage{amssymb}
\usepackage{pdfpages} % Gói để nhúng PDF
\usepackage{hyperref}
\usepackage{babel}
\usepackage{amsmath}
\usepackage{fancyhdr}
\usepackage{bigints}
\usepackage{tikz}
\usepackage{xcolor,listings}
\usepackage{tcolorbox} 
\pagestyle{fancy}
\usepackage{setspace}
\usepackage{hyperref}
\usepackage{enumitem}
\onehalfspacing % Giãn dòng 1.5 lần
\fancyhead[L]{\textbf{\textit{Principles of Database Management}}}
\fancyhead[R]{\textbf{\textit{Lab 4}}}
\fancyfoot[C]{\thepage} % Đánh số trang giữa
\setlength{\parindent}{0pt} % Lùi đầu dòng
\setlength{\parskip}{12pt} % Khoảng cách giữa các đoạn
\usepackage{array}
\newcommand{\query}[1]{
	\textbf{Query}: #1\par
	\textbf{SQL}:
}
\newcommand{\result}[1]{
	\textbf{Result set}:\par 
	\includegraphics[width = 0.45\textwidth]{#1}
}
\begin{document}
	\setstretch{1.25}
	\begin{titlepage}
		\begin{center}
			\Large{Trường Đại Học Quốc Tế - ĐHQG TP.HCM}
		\end{center}
		\vspace*{\fill}
		\begin{center}
			{\Huge{\textbf{LAB REPORT}}}\\[0.5cm]
			\date{-5ex}
			\Large{Course: Principles Of Database Management} \large{LAB 4}
		\end{center}
		\vspace*{\fill}
		\large{\textbf{Full Name}: Trần Minh Phúc}\dotfill\par
		\noindent \large{\textbf{Student's ID}: ITCSIU24070}\dotfill
	\end{titlepage}
	\section*{Basic Queries}
	\begin{enumerate}
		\item \query{Write a query to list the names of all employees.}
		\begin{lstlisting}[
			language=SQL,
			showspaces=false,
			basicstyle=\ttfamily,
			numbers=left,
			numberstyle=\tiny,
			commentstyle=\color{gray}
			]
SELECT 
  fname,
  minit,
  lname
FROM 
  employee;
		\end{lstlisting}
		\result{r1}
		\item \query{Write a query to list the names of all female employees.}
		\begin{lstlisting}[
			language=SQL,
			showspaces=false,
			basicstyle=\ttfamily,
			numbers=left,
			numberstyle=\tiny,
			commentstyle=\color{gray}
			]
SELECT 
  fname,
  minit,
  lname
FROM 
  employee
WHERE 
  sex = 'F';
		\end{lstlisting}
		\result{r2}
	\item \query{Write a query to list the names of all employees along with their department names.}
	\begin{lstlisting}[
		language=SQL,
		showspaces=false,
		basicstyle=\ttfamily,
		numbers=left,
		numberstyle=\tiny,
		commentstyle=\color{gray}
		]
SELECT 
  fname, 
  minit, 
  lname, 
  dname
FROM 
  employee,
  department
WHERE 
  employee.dno = department.dnumber; 		
	\end{lstlisting}
	\result{r3}
	\item \query{Write a query to list the names of all departments along with their manager’s social security number.}
	\begin{lstlisting}[
		language=SQL,
		showspaces=false,
		basicstyle=\ttfamily,
		numbers=left,
		numberstyle=\tiny,
		commentstyle=\color{gray}
		]
SELECT 
  dname, 
  mgrssn
FROM 
  department;		
	\end{lstlisting}
	\result{r4}	
	\end{enumerate}
	\section*{Aggregate Functions}
	\begin{enumerate}[start=5]
		\item \query{Write a query to find the average salary of employees in each department.}
		\begin{lstlisting}[
			language=SQL,
			showspaces=false,
			basicstyle=\ttfamily,
			numbers=left,
			numberstyle=\tiny,
			commentstyle=\color{gray}
			]
SELECT 
  dname,
  AVG(salary) AS avg_salary
FROM 
  department, 
  employee
WHERE 
  department.dnumber = employee.dno
GROUP BY 
  dname;			
		\end{lstlisting}
		\result{r5}
		\item \query{Write a query to list the names of employees who don’t work on any projects.}
		\begin{lstlisting}[
			language=SQL,
			showspaces=false,
			basicstyle=\ttfamily,
			numbers=left,
			numberstyle=\tiny,
			commentstyle=\color{gray}
			]
SELECT 
  fname,
  minit,
  lname
FROM 
  employee
WHERE 
  ssn NOT IN (
    SELECT 
      essn
    FROM 
      works_on
    );			
		\end{lstlisting}
		\result{r6}
	\item \query{Write a query to list the names of all employees and their dependents.}
	\begin{lstlisting}[
		language=SQL,
		showspaces=false,
		basicstyle=\ttfamily,
		numbers=left,
		numberstyle=\tiny,
		commentstyle=\color{gray}
		]
SELECT 
  fname,
  minit,
  lname,
  dependent_name
FROM 
  employee,
  dependent
WHERE 
  ssn = essn;   		
	\end{lstlisting}
	\result{r7}
	\item \query{Write a query to list the names of employees who are also managers of departments.}
	\begin{lstlisting}[
		language=SQL,
		showspaces=false,
		basicstyle=\ttfamily,
		numbers=left,
		numberstyle=\tiny,
		commentstyle=\color{gray}
		]
SELECT 
  fname,
  minit,
  lname
FROM
  employee,
  department
WHERE   
  ssn = mgrssn;		
	\end{lstlisting}
	\result{r8}
	\item \query{Write a query to find the names of all employees who work on every project.}
	\begin{lstlisting}[
		language=SQL,
		showspaces=false,
		basicstyle=\ttfamily,
		numbers=left,
		numberstyle=\tiny,
		commentstyle=\color{gray}
		]
SELECT 
  fname, 
  minit, 
  lname
FROM 
  employee AS e
WHERE 
  NOT EXISTS (
    SELECT 
      pnumber
    FROM 
      project AS p
    WHERE 
      NOT EXISTS (
        SELECT 
          w.essn
        FROM 
          works_on AS w
        WHERE 
          w.essn = e.ssn AND 
          w.pno = p.pnumber
      )
    );		
	\end{lstlisting}
	\result{r9}
	\item \query{Write a query to find the names of employees with the same supervisor as the employee with social security number ‘123456789’.}
	\begin{lstlisting}[
		language=SQL,
		showspaces=false,
		basicstyle=\ttfamily,
		numbers=left,
		numberstyle=\tiny,
		commentstyle=\color{gray}
		]
SELECT 
  fname,
  minit,
  lname
FROM
  employee
WHERE 
  superssn IN (
    SELECT 
      superssn
    FROM 
      employee
    WHERE 
      ssn = '123456789'
  );		
	\end{lstlisting}
	\result{r10}
	\end{enumerate}
	\section*{Complex Queries}
	\begin{enumerate}[start = 11]
		\item \query{Write a query to retrieve the names of all employees in department 5 who work more than 10 hours per week on the ‘ProductX’ project.
		}
		\begin{lstlisting}[
			language=SQL,
			showspaces=false,
			basicstyle=\ttfamily,
			numbers=left,
			numberstyle=\tiny,
			commentstyle=\color{gray}
			]
SELECT 
  fname,
  minit,
  lname
FROM 
  employee
WHERE 
  dno IN (
    SELECT 
      dnumber
    FROM 
      department
    WHERE
      dnumber = 5
    ) 
    AND ssn IN (
      SELECT 
        essn
      FROM 
        works_on
      WHERE
        hours > 10
      AND pno IN (
        SELECT 
          pno
        FROM 
          project
        WHERE
          pname = 'ProductX'
      )
    );			
		\end{lstlisting}
		\result{r11}
	\item \query{Write a query to list the names of all employees who have a dependent with the same first name as themselves.}
	\begin{lstlisting}[
		language=SQL,
		showspaces=false,
		basicstyle=\ttfamily,
		numbers=left,
		numberstyle=\tiny,
		commentstyle=\color{gray}
		]
SELECT 
  e.fname,
  e.minit,
  e.lname
FROM
  employee AS e
INNER JOIN dependent AS d ON e.ssn = d.essn
WHERE 
  e.fname = d.dependent_name; 		
	\end{lstlisting}
	\result{r12}
	\item \query{Write a query to find the names of all employees who are directly supervised by ‘Franklin Wong’.}
	\begin{lstlisting}[
		language=SQL,
		showspaces=false,
		basicstyle=\ttfamily,
		numbers=left,
		numberstyle=\tiny,
		commentstyle=\color{gray}
		]
SELECT 
  fname,
  minit,
  lname
FROM
  employee
WHERE superssn IN (
  SELECT 
    ssn
  FROM
    employee
  WHERE 
    fname = 'Frank'
    AND lname = 'Wong' 
  );		
	\end{lstlisting}
	\result{r13}
	\item \query{Write a query to list, for each project, the project name and the total hours per week spent on that project by all employees.}
	\begin{lstlisting}[
		language=SQL,
		showspaces=false,
		basicstyle=\ttfamily,
		numbers=left,
		numberstyle=\tiny,
		commentstyle=\color{gray}
		]
SELECT 
  pname,
  SUM(hours) AS total_hours
FROM works_on INNER JOIN works_on ON project.pnumber = works_on.pno
GROUP BY 
  pname;		
	\end{lstlisting}
	\result{r14}
	\item \query{Write a query to retrieve the names of all employees who work on every project.}
	\begin{lstlisting}[
		language=SQL,
		showspaces=false,
		basicstyle=\ttfamily,
		numbers=left,
		numberstyle=\tiny,
		commentstyle=\color{gray}
		]
SELECT 
  fname, 
  minit, 
  lname
FROM 
  employee AS e
WHERE 
  NOT EXISTS (
    SELECT 
      pnumber
    FROM 
      project AS p
    WHERE 
      NOT EXISTS (
        SELECT 
          w.essn
        FROM 
          works_on AS w
        WHERE 
          w.essn = e.ssn AND 
          w.pno = p.pnumber
      )
    );				
	\end{lstlisting}
	\result{r15}
	\item \query{Write a query to list the names of employees who do not work on any projects.}
	\begin{lstlisting}[
		language=SQL,
		showspaces=false,
		basicstyle=\ttfamily,
		numbers=left,
		numberstyle=\tiny,
		commentstyle=\color{gray}
		]
SELECT 
  fname,
  minit,
  lname
FROM 
  employee
WHERE 
  ssn NOT IN (
    SELECT 
      essn
    FROM 
      works_on
    );			
	\end{lstlisting}
	\result{r16}
	\end{enumerate}
	\section*{Advanced Queries}
	\begin{enumerate}[start = 17]
		\item \query{Write a query to retrieve the average salary of all female employees.}
		\begin{lstlisting}[
			language=SQL,
			showspaces=false,
			basicstyle=\ttfamily,
			numbers=left,
			numberstyle=\tiny,
			commentstyle=\color{gray}
			]
SELECT
  fname,
  minit,
  lname,
  AVG(salary)
FROM 
  employee
WHERE 
  sex = 'F'
GROUP BY   
  fname,
  minit,
  lname			
		\end{lstlisting}
		\result{r17}
	\item \query{Write a query to find the names and addresses of all employees who work on at least one project located in Houston but whose department has no location in Houston.}
	\begin{lstlisting}[
		language=SQL,
		showspaces=false,
		basicstyle=\ttfamily,
		numbers=left,
		numberstyle=\tiny,
		commentstyle=\color{gray}
		]
SELECT
  fname,
  minit,
  lname,
  address
FROM 
  employee
WHERE
  ssn IN (
    SELECT 
      essn
    FROM 
      works_on
    WHERE 
      pno IN (
        SELECT 
          pnumber
        FROM
          project
        WHERE
          plocation = 'Houston'
      )
    )
  AND dno NOT IN (
    SELECT 
      dnumber
    FROM 
      dept_locations
    WHERE
      dlocation = 'Houston'
  );		
	\end{lstlisting}
	\result{r18}
	\item \query{Write a query to list the last names of all department managers who have no dependents.}
	\begin{lstlisting}[
		language=SQL,
		showspaces=false,
		basicstyle=\ttfamily,
		numbers=left,
		numberstyle=\tiny,
		commentstyle=\color{gray}
		]
SELECT 
  lname
FROM 
  employee, 
  department 
WHERE 
  ssn = mgrssn
  AND ssn NOT IN (
    SELECT 
      essn
    FROM 
      dependent
  )
	\end{lstlisting}
	\result{r19}
	\item \query{Write a query to retrieve, for each department whose average employee salary is more than 30000, the department name and the number of employees working for that department.}
	\begin{lstlisting}[
		language=SQL,
		showspaces=false,
		basicstyle=\ttfamily,
		numbers=left,
		numberstyle=\tiny,
		commentstyle=\color{gray}
		]
SELECT
  dname,
  COUNT(ssn)
FROM 
  employee INNER JOIN department ON employee.dno = department.dnumber
GROUP BY 
  dname
HAVING 
  AVG(salary) > 30000 		
	\end{lstlisting}
	\result{r20}
	\end{enumerate}
	\begin{center}
		\begin{tikzpicture}
			\draw (-8,0) -- (-3.5,0);
			\node at (0,0) {\textit{This is the end of the report}};
			\draw (3.5,0) -- (8,0);
		\end{tikzpicture}
	\end{center}
\end{document}
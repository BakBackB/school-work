\documentclass[a4paper,12pt]{article} % Hoặc article nếu là sách ngắn
\usepackage[left=10mm, right=10mm, top=30mm, bottom=30mm]{geometry}
\usepackage[english, vietnamese]{babel}
\usepackage{amssymb}
\usepackage{pdfpages} % Gói để nhúng PDF
\usepackage{hyperref}
\usepackage{babel}
\usepackage{amsmath}
\usepackage{fancyhdr}
\usepackage{bigints}
\usepackage{tikz}
\usepackage{xcolor,listings}
\usepackage{tcolorbox} 
\pagestyle{fancy}
\usepackage{setspace}
\usepackage{hyperref}
\usepackage{enumitem}
\onehalfspacing % Giãn dòng 1.5 lần
\fancyhead[L]{\textbf{\textit{Principles of Database Management}}}
\fancyhead[R]{\textbf{\textit{Lab 5}}}
\fancyfoot[C]{\thepage} % Đánh số trang giữa
\setlength{\parindent}{0pt} % Lùi đầu dòng
\setlength{\parskip}{12pt} % Khoảng cách giữa các đoạn
\usepackage{array}
\newcommand{\query}[1]{
	\textbf{Query}: #1\par
	\textbf{SQL}:
}
\newcommand{\result}[1]{
	\textbf{Result set}:\par 
	\includegraphics[width = 0.65\textwidth]{#1}
}
\begin{document}
	\setstretch{1.25}
	\begin{titlepage}
		\begin{center}
			\Large{Trường Đại Học Quốc Tế - ĐHQG TP.HCM}
		\end{center}
		\vspace*{\fill}
		\begin{center}
			{\Huge{\textbf{LAB REPORT}}}\\[0.5cm]
			\date{-5ex}
			\Large{Course: Principles Of Database Management} \large{LAB 5}
		\end{center}
		\vspace*{\fill}
		\large{\textbf{Full Name}: Trần Minh Phúc}\dotfill\par
		\noindent \large{\textbf{Student's ID}: ITCSIU24070}\dotfill
	\end{titlepage}
	\begin{enumerate}
		\item \query{List how many complete years each staff member has been with the school as of October 1, 2017, and sort the result by last name and first name. (27 rows)}
		\begin{lstlisting}[
			language=SQL,
			showspaces=false,
			basicstyle=\ttfamily,
			numbers=left,
			numberstyle=\tiny,
			commentstyle=\color{gray}
			]
SELECT TOP 27
  StfFirstName,
  StfLastName,
  DATEDIFF(YEAR, DateHired, '2017-10-01') AS years_complete
FROM Staff;
		\end{lstlisting}
		\result{r1}
		\item \query{Show me a list of staff members, their salaries, and a proposed 7 percent bonus for each staff member. (27 rows)}
		\begin{lstlisting}[
			language=SQL,
			showspaces=false,
			basicstyle=\ttfamily,
			numbers=left,
			numberstyle=\tiny,
			commentstyle=\color{gray}
			]
SELECT TOP 27
  StfFirstName, 
  StfLastName,
  Salary,
  0.07 * Salary AS bonus
FROM Staff;
		\end{lstlisting}
		\result{r2}
		\item \query{Give me a list of staff members and show them in descending order of salary. (27 rows)}
		\begin{lstlisting}[
			language=SQL,
			showspaces=false,
			basicstyle=\ttfamily,
			numbers=left,
			numberstyle=\tiny,
			commentstyle=\color{gray}
			]
SELECT TOP 27
  StfFirstName,
  StfLastName,
  Salary
FROM Staff
ORDER BY Salary DESC;		
		\end{lstlisting}
		\result{r3}
		\item \query{Can you give me a staff member phone list? (27 rows)}
		\begin{lstlisting}[
			language=SQL,
			showspaces=false,
			basicstyle=\ttfamily,
			numbers=left,
			numberstyle=\tiny,
			commentstyle=\color{gray}
			]
SELECT TOP 27
  StfFirstName,
  StfLastName,
  StfPhoneNumber
FROM Staff;	
		\end{lstlisting}
		\result{r4}	
	\end{enumerate}
	\begin{enumerate}[start=5]
		\item \query{List the names of all our students and order them by the cities they live in. (18 rows)}
		\begin{lstlisting}[
			language=SQL,
			showspaces=false,
			basicstyle=\ttfamily,
			numbers=left,
			numberstyle=\tiny,
			commentstyle=\color{gray}
			]
SELECT TOP 18
  StudFirstName,
  StudLastName,
  StudCity
FROM Students
ORDER BY StudCity;		
		\end{lstlisting}
		\result{r5}
		\item \query{Show me an alphabetical list of all the staff members and their salaries if they make between \$40,000 and \$50,000 a year. (14 rows)}
		\begin{lstlisting}[
			language=SQL,
			showspaces=false,
			basicstyle=\ttfamily,
			numbers=left,
			numberstyle=\tiny,
			commentstyle=\color{gray}
			]
SELECT TOP 14
  StfFirstName,
  StfLastName,
  Salary
FROM Staff
WHERE Salary BETWEEN 40000 AND 50000;	
		\end{lstlisting}
		\result{r6}
		\item \query{Show me a list of students whose last name is ‘Kennedy’ or who live in Seattle. (4 rows)}
		\begin{lstlisting}[
			language=SQL,
			showspaces=false,
			basicstyle=\ttfamily,
			numbers=left,
			numberstyle=\tiny,
			commentstyle=\color{gray}
			]
SELECT TOP 4
  StudFirstName,
  StudLastName,
  StudCity
FROM Students
WHERE StudLastName = 'Kennedy'
  OR StudCity = 'Seattle';  		
		\end{lstlisting}
		\result{r7}
		\item \query{Show me which staff members use a post office box as their address. (3 rows).}
		\begin{lstlisting}[
			language=SQL,
			showspaces=false,
			basicstyle=\ttfamily,
			numbers=left,
			numberstyle=\tiny,
			commentstyle=\color{gray}
			]
SELECT TOP 3
  StfFirstName,
  StfLastName,
  StfStreetAddress
FROM Staff
WHERE StfStreetAddress LIKE '%Box%';	
		\end{lstlisting}
		\result{r8}
		\item \query{Can you show me which students live outside of the Pacific Northwest? (5 rows).}
		\begin{lstlisting}[
			language=SQL,
			showspaces=false,
			basicstyle=\ttfamily,
			numbers=left,
			numberstyle=\tiny,
			commentstyle=\color{gray}
			]
SELECT TOP 5
  StudFirstName,
  StudLastName,
  StudState
FROM Students
WHERE StudState NOT IN ('ID', 'OR', 'WA');
		\end{lstlisting}
		\result{r9}
		\item \query{List all the subjects that have a subject code starting ‘MUS’. (4 rows)}
		\begin{lstlisting}[
			language=SQL,
			showspaces=false,
			basicstyle=\ttfamily,
			numbers=left,
			numberstyle=\tiny,
			commentstyle=\color{gray}
			]
SELECT TOP 4
  SubjectName,
  SubjectCode
FROM Subjects
WHERE SubjectCode LIKE 'MUS%';
		\end{lstlisting}
		\result{r10}
	\end{enumerate}
	\begin{enumerate}[start = 11]
		\item \query{Produce a list of the ID numbers all the Associate Professors who are employed full time. (4 rows)}
		\begin{lstlisting}[
			language=SQL,
			showspaces=false,
			basicstyle=\ttfamily,
			numbers=left,
			numberstyle=\tiny,
			commentstyle=\color{gray}
			]
SELECT TOP 4
  StaffID,
  Title,
  Tenured
FROM Faculty
WHERE Title = 'Associate Professor'
  AND Status = 'Full Time';		
		\end{lstlisting}
		\result{r11}
		\item \query{List the subjects taught on Wednesday. SELECT DISTINCT}
		\begin{lstlisting}[
			language=SQL,
			showspaces=false,
			basicstyle=\ttfamily,
			numbers=left,
			numberstyle=\tiny,
			commentstyle=\color{gray}
			]
SELECT DISTINCT TOP 4
  SubjectName
FROM Subjects
WHERE SubjectID IN (
  SELECT SubjectID
  FROM Classes
  WHERE WednesdaySchedule > 0
);		
		\end{lstlisting}
		\result{r12}
		\item \query{Show me the students and teachers who have the same first name.}
		\begin{lstlisting}[
			language=SQL,
			showspaces=false,
			basicstyle=\ttfamily,
			numbers=left,
			numberstyle=\tiny,
			commentstyle=\color{gray}
			]
SELECT
  StudentID,
  StudFirstName,
  StudLastName,
  StaffID,
  StfFirstName,
  StfLastName
FROM Students, Staff
WHERE StudFirstName = StfFirstName;
		\end{lstlisting}
		\result{r13}
		\item \query{Display buildings and all the classrooms in each building. (47 rows)}
		\begin{lstlisting}[
			language=SQL,
			showspaces=false,
			basicstyle=\ttfamily,
			numbers=left,
			numberstyle=\tiny,
			commentstyle=\color{gray}
			]
SELECT TOP 47
  c_r.BuildingCode,
  b.BuildingName,
  ClassRoomID
FROM Class_Rooms AS c_r,
  Buildings AS b
WHERE c_r.BuildingCode = b.BuildingCode;
		\end{lstlisting}
		\result{r14}
		\item \query{List students and all the classes in which they are currently enrolled. (50 rows)}
		\begin{lstlisting}[
			language=SQL,
			showspaces=false,
			basicstyle=\ttfamily,
			numbers=left,
			numberstyle=\tiny,
			commentstyle=\color{gray}
			]
SELECT TOP 50
  s.StudentID,
  StudFirstName,
  StudLastName,
  ClassID
FROM 
  Student_Schedules AS s_s, 
  Students AS s
WHERE s_s.StudentID = s.StudentID
  AND ClassStatus IN(
    SELECT ClassStatus
    FROM Student_Class_Status
    WHERE ClassStatusDescription = 'Enrolled'
  );	
		\end{lstlisting}
		\result{r15}
		\item \query{List the faculty staff and the subject each teaches. (110 rows)}
		\begin{lstlisting}[
			language=SQL,
			showspaces=false,
			basicstyle=\ttfamily,
			numbers=left,
			numberstyle=\tiny,
			commentstyle=\color{gray}
			]
SELECT TOP 110
  s.StaffID,
  StfFirstName,
  StfLastName,
  SubjectName
FROM Faculty_Subjects AS f_sub,
  Staff AS s,
  Subjects AS sub
WHERE f_sub.StaffID = s.StaffID
  AND f_sub.SubjectID = sub.SubjectID;	
		\end{lstlisting}
		\result{r16}
	\end{enumerate}
	\begin{enumerate}[start = 17]
		\item \query{Show me the students who have a grade of 85 or better in art and who also have a grade of 85 or better in any computer course. (1 row)}
		\begin{lstlisting}[
			language=SQL,
			showspaces=false,
			basicstyle=\ttfamily,
			numbers=left,
			numberstyle=\tiny,
			commentstyle=\color{gray}
			]
SELECT TOP 1
  s.StudentID,
  StudFirstName,
  StudLastName,
  Grade
FROM Students AS s INNER JOIN Student_Schedules AS s_s
  ON s.StudentID = s_s.StudentID
WHERE Grade >= 85
  AND ClassID IN (
    SELECT ClassID
    FROM Classes
    WHERE SubjectID IN (
      SELECT SubjectID
      FROM Subjects
      WHERE CategoryID IN (
        SELECT CategoryID
        FROM Categories
        WHERE CategoryDescription IN ('Art', 'Computer Science')
      )
    )
  );		
		\end{lstlisting}
		\result{r17}
		\item \query{Show me classes that have no students enrolled. (Hint: You need only enrolled rows from Student\_Classes, not completed or withdrew.) (118 rows)}
		\begin{lstlisting}[
			language=SQL,
			showspaces=false,
			basicstyle=\ttfamily,
			numbers=left,
			numberstyle=\tiny,
			commentstyle=\color{gray}
			]
SELECT TOP 118
  ClassID,
  ClassStatus
FROM Student_Schedules
WHERE ClassStatus NOT IN (
  SELECT ClassStatus
  FROM Student_Class_Status
  WHERE ClassStatusDescription = 'Enrolled'
);
		\end{lstlisting}
		\result{r18}
		\item \query{Display subjects with no faculty assigned. (1 row)}
		\begin{lstlisting}[
			language=SQL,
			showspaces=false,
			basicstyle=\ttfamily,
			numbers=left,
			numberstyle=\tiny,
			commentstyle=\color{gray}
			]
SELECT TOP 1 SubjectName
FROM Subjects
WHERE SubjectID NOT IN (
  SELECT SubjectID
  FROM Faculty_Subjects
);
		\end{lstlisting}
		\result{r19}
		\item \query{List students not currently enrolled in any classes. (Hint: You need to find which students have an enrolled class status in student schedules and then find the students who are not
			in this set.) (2 rows)}
		\begin{lstlisting}[
			language=SQL,
			showspaces=false,
			basicstyle=\ttfamily,
			numbers=left,
			numberstyle=\tiny,
			commentstyle=\color{gray}
			]
SELECT TOP 2
  StudentID,
  StudFirstName,
  StudLastName
FROM Students
WHERE StudentID IN (
  SELECT StudentID
  FROM Student_Schedules
  WHERE ClassStatus NOT IN (
    SELECT ClassStatus
    FROM Student_Class_Status
    WHERE ClassStatusDescription = 'Enrolled'
  )
);
		\end{lstlisting}
		\result{r20}
	\item \query{Display all faculty and the classes they are scheduled to teach. (135 rows)}
	\begin{lstlisting}[
		language=SQL,
		showspaces=false,
		basicstyle=\ttfamily,
		numbers=left,
		numberstyle=\tiny,
		commentstyle=\color{gray}
		]
SELECT TOP 135
  s.StaffID,
  Title,
  StfFirstName,
  StfLastname,
  ClassID
FROM Faculty AS f INNER JOIN Staff s 
  ON f.StaffID = s.StaffID
  INNER JOIN Faculty_Classes AS f_c 
  ON s.StaffID = f_c.StaffID;		
	\end{lstlisting}
	\result{r21}
	\item \query{Show me the students who have a grade of 85 or better in Art together with the faculty	members who teach Art and have a proficiency rating of 9 or better. (12 rows)}
	\begin{lstlisting}[
		language=SQL,
		showspaces=false,
		basicstyle=\ttfamily,
		numbers=left,
		numberstyle=\tiny,
		commentstyle=\color{gray}
		]
WITH Art_Faculty AS (
SELECT 
  s.StaffID,
  StfFirstName,
  StfLastName,
  SubjectID,
  ProficiencyRating
FROM Staff AS s INNER JOIN Faculty_Subjects AS f_sub
  ON s.StaffID = f_sub.StaffID
WHERE ProficiencyRating >= 9
  AND SubjectID IN (
    SELECT SubjectID
    FROM Subjects
    WHERE CategoryID IN (
      SELECT CategoryID
      FROM Categories
      WHERE CategoryDescription = 'Art'
    )
  )
), Art_Student AS (
SELECT
  s_s.StudentID,
  StudFirstName,
  StudLastName,
  s_s.ClassID,
  SubjectID,
  Grade
FROM Student_Schedules AS s_s INNER JOIN Classes AS c
  ON s_s.ClassID = c.ClassID
  INNER JOIN Students AS s
  ON s_s.StudentID = s.StudentID
WHERE Grade >= 85 
  AND SubjectID IN (
    SELECT SubjectID
    FROM Subjects
    WHERE CategoryID IN (
      SELECT CategoryID
      FROM Categories
      WHERE CategoryDescription = 'Art'
    )
  )
)
SELECT TOP 12
  StudentID,
  StudFirstName,
  StudLastName,
  a_s.SubjectID,
  Grade,
  StaffID,
  StfFirstName,
  StfLastname,
  ProficiencyRating
FROM Art_Faculty AS a_f LEFT JOIN Art_Student AS a_s
  ON a_f.SubjectID = a_s.SubjectID;		
	\end{lstlisting}
	\result{r22}
	
	\end{enumerate}
	\begin{center}
		\begin{tikzpicture}
			\draw (-8,0) -- (-3.5,0);
			\node at (0,0) {\textit{This is the end of the report}};
			\draw (3.5,0) -- (8,0);
		\end{tikzpicture}
	\end{center}
\end{document}
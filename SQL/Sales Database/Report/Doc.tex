\documentclass[a4paper,12pt]{article} % Hoặc article nếu là sách ngắn
\usepackage[left=10mm, right=10mm, top=30mm, bottom=30mm]{geometry}
\usepackage[english, vietnamese]{babel}
\usepackage{amssymb}
\usepackage{pdfpages} % Gói để nhúng PDF
\usepackage{hyperref}
\usepackage{babel}
\usepackage{amsmath}
\usepackage{fancyhdr}
\usepackage{bigints}
\usepackage{tikz}
\usepackage{xcolor,listings}
\usepackage{tcolorbox} 
\pagestyle{fancy}
\usepackage{setspace}
\usepackage{hyperref}
\usepackage{enumitem}
\onehalfspacing % Giãn dòng 1.5 lần
\fancyhead[L]{\textbf{\textit{Principles of Database Management}}}
\fancyhead[R]{\textbf{\textit{Lab 8}}}
\fancyfoot[C]{\thepage} % Đánh số trang giữa
\setlength{\parindent}{0pt} % Lùi đầu dòng
\setlength{\parskip}{12pt} % Khoảng cách giữa các đoạn
\definecolor{backcolour}{RGB}{246,246,246}
\definecolor{codepurple}{RGB}{0,206,209}
\usepackage{array}
\newcommand{\query}[1]{
	\textbf{Query}: #1\par
	\textbf{SQL}:
}
\newcommand{\result}[1]{
	\textbf{Result set}:\par 
	\includegraphics[width = 0.9425\textwidth]{#1}
}
\begin{document}
	\setstretch{1.25}
	\begin{titlepage}
		\begin{center}
			\Large{Trường Đại Học Quốc Tế - ĐHQG TP.HCM}
		\end{center}
		\vspace*{\fill}
		\begin{center}
			{\Huge{\textbf{LAB REPORT}}}\\[0.5cm]
			\date{-5ex}
			\Large{Course: Principles Of Database Management} \large{8}
		\end{center}
		\vspace*{\fill}
		\large{\textbf{Full Name}: Trần Minh Phúc}\dotfill\par
		\noindent \large{\textbf{Student's ID}: ITCSIU24070}\dotfill
	\end{titlepage}
	\begin{enumerate}
		\item \query{Show me all the orders shipped on October 3, 2017, and each order's related customer last name.}
		\begin{lstlisting}[
			backgroundcolor=\color{backcolour}, 
			language=SQL,
			keywordstyle=\color{codepurple},
			showspaces=false,
			basicstyle=\ttfamily,
			numbers=left,
			numberstyle=\tiny,
			commentstyle=\color{gray}
			]
SELECT
  OrderNumber,
  CustomerID,
  (SELECT CustLastName
    FROM Customers
    WHERE CustomerID = o.CustomerID
  ) AS CustomerLastName
FROM Orders AS o
WHERE ShipDate = '2017-10-3';
GO		
	\end{lstlisting}
		\result{r1}
		\item \query{List all the customer names and a count of the orders they placed.}
		\begin{lstlisting}[
			backgroundcolor=\color{backcolour}, 
			language=SQL,
			keywordstyle=\color{codepurple},
			showspaces=false,
			basicstyle=\ttfamily,
			numbers=left,
			numberstyle=\tiny,
			commentstyle=\color{gray}
			]
SELECT
  CustomerID,
  CustFirstName,
  CustLastName,
  (SELECT
    COUNT(OrderNumber)
    FROM Orders
    WHERE c.CustomerID = CustomerID
    GROUP BY CustomerID
  ) AS OrdersPlaced
FROM Customers AS c;
GO
		\end{lstlisting}
		\result{r2}
		\item \query{List customers and all the details from their last order.}
		\begin{lstlisting}[
			backgroundcolor=\color{backcolour}, 
			language=SQL,
			keywordstyle=\color{codepurple},
			showspaces=false,
			basicstyle=\ttfamily,
			numbers=left,
			numberstyle=\tiny,
			commentstyle=\color{gray}
			]
SELECT
  c.CustomerID,
  c.CustFirstName,
  c.CustLastName,
  lo.OrderNumber,
  lo.OrderDate AS LastOrderDate,
  d.ProductNumber,
  d.QuotedPrice,
  d.QuantityOrdered
FROM Customers AS c
LEFT JOIN (
  SELECT o.CustomerID, o.OrderNumber, o.OrderDate
  FROM Orders o
  WHERE o.OrderDate = (
    SELECT MAX(OrderDate)
    FROM Orders o2
    WHERE o2.CustomerID = o.CustomerID
  )
) AS lo
ON lo.CustomerID = c.CustomerID
LEFT JOIN Order_Details AS d
ON d.OrderNumber = lo.OrderNumber;
GO	
		\end{lstlisting}
		\result{r3}
		\item \query{Find all accessories that are priced greater than any clothing item. (hint use ALL)}
		\begin{lstlisting}[
			backgroundcolor=\color{backcolour}, 
			language=SQL,
			keywordstyle=\color{codepurple},
			showspaces=false,
			basicstyle=\ttfamily,
			numbers=left,
			numberstyle=\tiny,
			commentstyle=\color{gray}
			]
SELECT
  ProductNumber,
  ProductName,
  RetailPrice
FROM Products
WHERE CategoryID IN (
  SELECT CategoryID
  FROM Categories
  WHERE CategoryDescription = 'Accessories'
)
AND RetailPrice >  ALL(
  SELECT
  RetailPrice
  FROM Products
  WHERE CategoryID IN (
    SELECT CategoryID
    FROM Categories
    WHERE CategoryDescription = 'Clothing'
  )
);
GO
		\end{lstlisting}
		\result{r4}
		\item \query{Find all the customers who ordered a bicycle. (Use EXISTS)}
		\begin{lstlisting}[
			backgroundcolor=\color{backcolour}, 
			language=SQL,
			keywordstyle=\color{codepurple},
			showspaces=false,
			basicstyle=\ttfamily,
			numbers=left,
			numberstyle=\tiny,
			commentstyle=\color{gray}
			]
SELECT
  c.CustomerID AS Bicycle_CustomerID,
  c.CustFirstName,
  c.CustLastName
FROM Customers AS c
WHERE EXISTS (
  SELECT 1
  FROM Orders AS o
  WHERE c.CustomerID = o.CustomerID
  AND EXISTS (
    SELECT 1
    FROM Order_Details AS od
    WHERE o.OrderNumber = od.OrderNumber
    AND EXISTS (
      SELECT 1
      FROM Products AS p
      WHERE od.ProductNumber = p.ProductNumber
      AND EXISTS (
        SELECT 1
        FROM Categories AS cat
        WHERE p.CategoryID = cat.CategoryID
        AND CategoryDescription = 'Bikes'
      )
    )
  )
);
GO	
		\end{lstlisting}
		\result{r5}
		\item \query{List customers who ordered bikes.(use IN)}
		\begin{lstlisting}[
			backgroundcolor=\color{backcolour}, 
			language=SQL,
			keywordstyle=\color{codepurple},
			showspaces=false,
			basicstyle=\ttfamily,
			numbers=left,
			numberstyle=\tiny,
			commentstyle=\color{gray}
			]
SELECT
  CustomerID AS Bicycle_CustomerID,
  CustFirstName,
  CustLastName
FROM Customers
WHERE CustomerID IN (
  SELECT CustomerID
  FROM Orders
  WHERE OrderNumber IN (
    SELECT OrderNumber
    FROM Order_Details
    WHERE ProductNumber IN (
      SELECT ProductNumber
      FROM Products
      WHERE CategoryID IN (
        SELECT CategoryID
        FROM Categories AS cat
        WHERE CategoryDescription = 'Bikes'
      )
    )
  )
);
GO	
		\end{lstlisting}
		\result{r6}
		\item \query{Display customers who ordered clothing or accessories.}
		\begin{lstlisting}[
			backgroundcolor=\color{backcolour}, 
			language=SQL,
			keywordstyle=\color{codepurple},
			showspaces=false,
			basicstyle=\ttfamily,
			numbers=left,
			numberstyle=\tiny,
			commentstyle=\color{gray}
			]
SELECT
  CustomerID AS Cloth_Acces_CusID,
  CustFirstName,
  CustLastName
FROM Customers
WHERE CustomerID = SOME (
  SELECT CustomerID
  FROM Orders
  WHERE OrderNumber = SOME (
    SELECT OrderNumber
    FROM Order_Details
    WHERE ProductNumber = SOME (
      SELECT ProductNumber
      FROM Products
      WHERE CategoryID = SOME (
        SELECT CategoryID
        FROM Categories
        WHERE CategoryDescription = 'Clothing' 
        OR CategoryDescription = 'Accessories'
      )
    )
  )
);
GO
		\end{lstlisting}
		\result{r7}
		\item \query{Find all the customers who ordered a bicycle helmet.}
		\begin{lstlisting}[
			backgroundcolor=\color{backcolour}, 
			language=SQL,
			keywordstyle=\color{codepurple},
			showspaces=false,
			basicstyle=\ttfamily,
			numbers=left,
			numberstyle=\tiny,
			commentstyle=\color{gray}
			]
SELECT
  CustomerID AS Bicycle_Helmet_CusID,
  CustFirstName,
  CustLastName
FROM Customers
WHERE CustomerID IN (
  SELECT CustomerID
  FROM Orders
  WHERE OrderNumber IN (
    SELECT OrderNumber
    FROM Order_Details
    WHERE ProductNumber IN (	
      SELECT ProductNumber
      FROM Products
      WHERE ProductName LIKE '%Helmet%'
    )
  )
);
GO
		\end{lstlisting}
		\result{r8}
	\item \query{What products have never been ordered?}
	\begin{lstlisting}[
		backgroundcolor=\color{backcolour}, 
		language=SQL,
		keywordstyle=\color{codepurple},
		showspaces=false,
		basicstyle=\ttfamily,
		numbers=left,
		numberstyle=\tiny,
		commentstyle=\color{gray}
		]
SELECT
  ProductNumber,
  ProductName AS Not_Ordered_Product
FROM Products
WHERE ProductNumber NOT IN (
  SELECT ProductNumber
  FROM Order_Details
);
GO		
	\end{lstlisting}
	\result{r9}
	\item \query{List vendors and a count of the products they sell to us.
	}
	\begin{lstlisting}[
		backgroundcolor=\color{backcolour}, 
		language=SQL,
		keywordstyle=\color{codepurple},
		showspaces=false,
		basicstyle=\ttfamily,
		numbers=left,
		numberstyle=\tiny,
		commentstyle=\color{gray}
		]
SELECT 
  VendorID,
  (SELECT VendName
    FROM Vendors
    WHERE Vendors.VendorID = Product_Vendors.VendorID
  ) AS VendName,
  COUNT(ProductNumber) AS Sold_To_Us
FROM Product_Vendors
GROUP BY VendorID;
GO		
	\end{lstlisting}
	\result{r10}
	\item \query{Display customers who ordered clothing or accessories}
	\begin{lstlisting}[
		backgroundcolor=\color{backcolour}, 
		language=SQL,
		keywordstyle=\color{codepurple},
		showspaces=false,
		basicstyle=\ttfamily,
		numbers=left,
		numberstyle=\tiny,
		commentstyle=\color{gray}
		]
SELECT
  CustomerID AS Cloth_Acces_CusID,
  CustFirstName,
  CustLastName
FROM Customers
WHERE CustomerID = ANY (
  SELECT CustomerID
  FROM Orders
  WHERE OrderNumber = ANY (
    SELECT OrderNumber
    FROM Order_Details
    WHERE ProductNumber = ANY (
      SELECT ProductNumber
      FROM Products
      WHERE CategoryID = ANY (
        SELECT CategoryID
        FROM Categories
        WHERE CategoryDescription = 'Clothing'
        OR CategoryDescription = 'Accessories'
      )
    )
  )
);
GO		
	\end{lstlisting}
	\result{r11}
	\item \query{Display products and the latest date each product was ordered. (Hint: Use the MAX aggregate function.) (40 rows).}
	\begin{lstlisting}[
		backgroundcolor=\color{backcolour}, 
		language=SQL,
		keywordstyle=\color{codepurple},
		showspaces=false,
		basicstyle=\ttfamily,
		numbers=left,
		numberstyle=\tiny,
		commentstyle=\color{gray}
		]
SELECT TOP 40
  ProductName,
  (SELECT MAX(OrderDate)
    FROM Orders AS o
    WHERE o.OrderNumber IN (
      SELECT OrderNumber
      FROM Order_Details AS od
      WHERE od.ProductNumber = p.ProductNumber
    )
  ) AS Latest_Date
FROM Products AS p;
GO		
	\end{lstlisting}
	\result{r12}
	\item \query{Calculate a total of all unique wholesale costs for the products we sell. (use SUM)}
	\begin{lstlisting}[
		backgroundcolor=\color{backcolour}, 
		language=SQL,
		keywordstyle=\color{codepurple},
		showspaces=false,
		basicstyle=\ttfamily,
		numbers=left,
		numberstyle=\tiny,
		commentstyle=\color{gray}
		]
SELECT SUM(DISTINCT WholesalePrice) AS Total_Cost
FROM Product_Vendors;
GO		
	\end{lstlisting}
	\result{r13}
	\item \query{What is the average item total for order 64?}
	\begin{lstlisting}[
		backgroundcolor=\color{backcolour}, 
		language=SQL,
		keywordstyle=\color{codepurple},
		showspaces=false,
		basicstyle=\ttfamily,
		numbers=left,
		numberstyle=\tiny,
		commentstyle=\color{gray}
		]
SELECT AVG(QuotedPrice * QuantityOrdered) AS Average_Total
FROM Order_Details
WHERE OrderNumber = 64;
GO		
	\end{lstlisting}
	\result{r14}	
	\item \query{Calculate an average of all unique product prices.}
\begin{lstlisting}[
	backgroundcolor=\color{backcolour}, 
	language=SQL,
	keywordstyle=\color{codepurple},
	showspaces=false,
	basicstyle=\ttfamily,
	numbers=left,
	numberstyle=\tiny,
	commentstyle=\color{gray}
	]
SELECT AVG(DISTINCT RetailPrice) AS Average_Product_Price
FROM Products;
GO	
\end{lstlisting}
\result{r15}
\item \query{What is the lowest price we charge for a product?}
\begin{lstlisting}[
	backgroundcolor=\color{backcolour}, 
	language=SQL,
	keywordstyle=\color{codepurple},
	showspaces=false,
	basicstyle=\ttfamily,
	numbers=left,
	numberstyle=\tiny,
	commentstyle=\color{gray}
	]
SELECT MIN(RetailPrice) AS Lowest_Product_Price
FROM Products;
GO	
\end{lstlisting}
\result{r16}

\item \query{How many different products were ordered on order number 553, and what was the total cost of that order? (use SUM and COUNT)}
\begin{lstlisting}[
	backgroundcolor=\color{backcolour}, 
	language=SQL,
	keywordstyle=\color{codepurple},
	showspaces=false,
	basicstyle=\ttfamily,
	numbers=left,
	numberstyle=\tiny,
	commentstyle=\color{gray}
	]
SELECT
  OrderNumber,
  COUNT(Ordernumber) AS Number_Ordered,
  SUM(QuotedPrice * QuantityOrdered) AS Total_Price
FROM Order_Details
GROUP BY OrderNumber
HAVING OrderNumber = 553;
GO

\end{lstlisting}
\result{r17}
\item \query{List the product names and numbers that have a quoted price greater than or equal to the overall average retail price in the products table.}
\begin{lstlisting}[
	backgroundcolor=\color{backcolour}, 
	language=SQL,
	keywordstyle=\color{codepurple},
	showspaces=false,
	basicstyle=\ttfamily,
	numbers=left,
	numberstyle=\tiny,
	commentstyle=\color{gray}
	]
SELECT
  p.ProductNumber,
  p.ProductName,
  od.QuotedPrice,
  (SELECT AVG(RetailPrice)
  FROM Products) AS Average_Retail
FROM Products AS p INNER JOIN Order_Details AS od
ON p.ProductNumber = od.ProductNumber
WHERE QuotedPrice >= (
  SELECT AVG(RetailPrice)
  FROM Products
);
GO	
\end{lstlisting}
\result{r18}
\item \query{What is the average retail price of a mountain bike?}
\begin{lstlisting}[
	backgroundcolor=\color{backcolour}, 
	language=SQL,
	keywordstyle=\color{codepurple},
	showspaces=false,
	basicstyle=\ttfamily,
	numbers=left,
	numberstyle=\tiny,
	commentstyle=\color{gray}
	]
SELECT
  ProductNumber,
  ProductName,
  AVG(RetailPrice) AS Average_Retail_Price
FROM Products
WHERE CategoryID IN (
  SELECT CategoryID
  FROM Categories
  WHERE CategoryDescription = 'Bikes'
)
GROUP BY ProductNumber, ProductName
ORDER BY Average_Retail_Price;
GO	
\end{lstlisting}
\result{r19}
\item \query{What was the date of our most recent order?}
\begin{lstlisting}[
	backgroundcolor=\color{backcolour}, 
	language=SQL,
	keywordstyle=\color{codepurple},
	showspaces=false,
	basicstyle=\ttfamily,
	numbers=left,
	numberstyle=\tiny,
	commentstyle=\color{gray}
	]
SELECT 
  OrderNumber,
  MAX(OrderDate) AS Latest_Order_Date
FROM Orders
GROUP BY OrderNumber;
GO
\end{lstlisting}
\result{r20}
\item \query{What was the total amount for order number 8?}
\begin{lstlisting}[
	backgroundcolor=\color{backcolour}, 
	language=SQL,
	keywordstyle=\color{codepurple},
	showspaces=false,
	basicstyle=\ttfamily,
	numbers=left,
	numberstyle=\tiny,
	commentstyle=\color{gray}
	]
SELECT COUNT(OrderNumber) AS Total_Order
FROM Order_Details
WHERE OrderNumber = 8;
GO		
\end{lstlisting}
\result{r21}
\item \query{Show me each vendor and the average by vendor of the number of days to deliver products. (Hint: Use the AVG aggregate function and group on vendor.)}
\begin{lstlisting}[
	backgroundcolor=\color{backcolour}, 
	language=SQL,
	keywordstyle=\color{codepurple},
	showspaces=false,
	basicstyle=\ttfamily,
	numbers=left,
	numberstyle=\tiny,
	commentstyle=\color{gray}
	]
SELECT 
  v.VendorID,
  v.VendName,
  AVG(pv.DaysToDeliver) AS Average_Deliver
FROM Vendors AS v INNER JOIN Product_Vendors AS pv
ON v.VendorID = pv.VendorID
GROUP BY v.VendorID, v.VendName;
GO	
\end{lstlisting}
\result{r22}
\item \query{My clothing supplier just announced a price increase of 4 percent. Update the price of the clothing products and add 4 percent. (use UPDATE)}
\begin{lstlisting}[
	backgroundcolor=\color{backcolour}, 
	language=SQL,
	keywordstyle=\color{codepurple},
	showspaces=false,
	basicstyle=\ttfamily,
	numbers=left,
	numberstyle=\tiny,
	commentstyle=\color{gray}
	]	
UPDATE Product_Vendors
SET WholesalePrice = 1.04*WholesalePrice
WHERE ProductNumber IN (
  SELECT ProductNumber
  FROM Products
  WHERE CategoryID IN (
    SELECT CategoryID
    FROM Categories
    WHERE CategoryDescription = 'Clothing'
  )
);
GO	
\end{lstlisting}
\item \query{Modify products by increasing the retail price by 4 percent for products that are clothing. (use UPDATE)}
\begin{lstlisting}[
	backgroundcolor=\color{backcolour}, 
	language=SQL,
	keywordstyle=\color{codepurple},
	showspaces=false,
	basicstyle=\ttfamily,
	numbers=left,
	numberstyle=\tiny,
	commentstyle=\color{gray}
	]
UPDATE Products
SET RetailPrice = RetailPrice + 0.04*RetailPrice
WHERE CategoryID IN (
  SELECT CategoryID
  FROM Categories
  WHERE CategoryDescription = 'Clothing'
);
GO	
\end{lstlisting}
\item \query{Change the orders table by setting the order total to the sum of quantity ordered times quoted price for all related order detail rows. (use UPDATE with subquery)}
\begin{lstlisting}[
	backgroundcolor=\color{backcolour}, 
	language=SQL,
	keywordstyle=\color{codepurple},
	showspaces=false,
	basicstyle=\ttfamily,
	numbers=left,
	numberstyle=\tiny,
	commentstyle=\color{gray}
	]
UPDATE Orders
SET OrderTotal = sub.OrderTotal
FROM Orders AS o INNER JOIN (
  SELECT OrderNumber, SUM(QuotedPrice*QuantityOrdered) AS OrderTotal
  FROM Order_Details AS od
  GROUP BY OrderNumber
) AS sub ON o.OrderNumber = sub.OrderNumber;
GO	
\end{lstlisting}
\item \query{Reduce the quoted price by 2 percent for orders shipped more than 30 days after the order date. (use UPDATE with subquery)}
\begin{lstlisting}[
	backgroundcolor=\color{backcolour}, 
	language=SQL,
	keywordstyle=\color{codepurple},
	showspaces=false,
	basicstyle=\ttfamily,
	numbers=left,
	numberstyle=\tiny,
	commentstyle=\color{gray}
	]
UPDATE Order_Details
SET QuotedPrice = 0.98*QuotedPrice
WHERE OrderNumber IN (
  SELECT OrderNumber
  FROM Orders
  WHERE DATEDIFF(DAY, OrderDate, ShipDate) > 30
);
GO	
\end{lstlisting}
\item \query{* Make sure the retail price for all bikes is at least a 45 percent markup over the wholesale price of the vendor with the lowest cost. (update and subsequent) - 1 row}
\begin{lstlisting}[
	backgroundcolor=\color{backcolour}, 
	language=SQL,
	keywordstyle=\color{codepurple},
	showspaces=false,
	basicstyle=\ttfamily,
	numbers=left,
	numberstyle=\tiny,
	commentstyle=\color{gray}
	]
UPDATE Products
SET RetailPrice = 1.45 * WholesalePrice
FROM Products AS p INNER JOIN Product_Vendors AS pv
ON p.ProductNumber = pv.ProductNumber
WHERE WholesalePrice = (
  SELECT MIN(WholesalePrice)
  FROM Product_Vendors)
AND RetailPrice < (
SELECT MIN(WholesalePrice) * 1.45
FROM Product_Vendors
)
AND CategoryID IN (
  SELECT CategoryID
  FROM Categories
  WHERE CategoryDescription = 'Bikes'
);
GO	
\end{lstlisting}
\item \query{Apply a 5 percent discount to all orders for customers who purchased more than
	\$50,000 in the month of October 2017. (hint You need a subquery within a subquery to
	fetch the order numbers for all orders where the customer ID of the order is in the set of
	customers who ordered more than \$50,000 in the month of October.) (639 rows changed)}
\begin{lstlisting}[
	backgroundcolor=\color{backcolour}, 
	language=SQL,
	keywordstyle=\color{codepurple},
	showspaces=false,
	basicstyle=\ttfamily,
	numbers=left,
	numberstyle=\tiny,
	commentstyle=\color{gray}
	]
UPDATE Orders
SET OrderTotal = 0.95 * OrderTotal
WHERE OrderNumber IN (
  SELECT OrderNumber
  FROM Orders
  WHERE CustomerID IN (
    SELECT CustomerID
    FROM Orders
    WHERE MONTH(OrderDate) = 10 AND YEAR(OrderDate) = 2017
    GROUP BY CustomerID
    HAVING SUM(OrderTotal) > 50000
  )
);
GO	
\end{lstlisting}
\item \query{Set the retail price of accessories (category = 1) to the wholesale price of the highestpriced vendor plus 35 percent. (11 rows changed).}
\begin{lstlisting}[
	backgroundcolor=\color{backcolour}, 
	language=SQL,
	keywordstyle=\color{codepurple},
	showspaces=false,
	basicstyle=\ttfamily,
	numbers=left,
	numberstyle=\tiny,
	commentstyle=\color{gray}
	]
UPDATE Products
SET RetailPrice = (
  SELECT MAX(WholesalePrice) * 1.35
  FROM Product_Vendors
)
WHERE CategoryID = 1;
GO	
\end{lstlisting}
\item \query{Copy to the Employees table the relevant columns in the Customers table for customer David Smith. (INSERT INTO)}
\begin{lstlisting}[
	backgroundcolor=\color{backcolour}, 
	language=SQL,
	keywordstyle=\color{codepurple},
	showspaces=false,
	basicstyle=\ttfamily,
	numbers=left,
	numberstyle=\tiny,
	commentstyle=\color{gray}
	]
INSERT INTO Employees(EmpFirstName, EmpLastName, EmpStreetAddress, Em
pCity, EmpState, EmpZipCode, EmpAreaCode, EmpPhoneNumber)
SELECT CustFirstName, CustLastName, CustStreetAddress, CustCity, Cust
State, CustZipCode, CustAreaCode, CustPhoneNumber
FROM Customers
WHERE CustFirstName = 'David' AND CustLastName = 'Smith';
GO
	
\end{lstlisting}
\item \query{Add a new product named 'Hot Dog Spinner' with a retail price of \$895 in the Bikes category. (INSERT INTO)}
\begin{lstlisting}[
	backgroundcolor=\color{backcolour}, 
	language=SQL,
	keywordstyle=\color{codepurple},
	showspaces=false,
	basicstyle=\ttfamily,
	numbers=left,
	numberstyle=\tiny,
	commentstyle=\color{gray}
	]
INSERT INTO Products(ProductName, RetailPrice, CategoryID)
VALUES('Hot Dog Spinner', 895, (SELECT CategoryID FROM Categories 
WHERE CategoryDescription = 'Bikes'));
GO	
\end{lstlisting}
	\end{enumerate}
	\begin{center}
		\begin{tikzpicture}
			\draw (-8,0) -- (-3.5,0);
			\node at (0,0) {\textit{This is the end of the report}};
			\draw (3.5,0) -- (8,0);
		\end{tikzpicture}
	\end{center}
\end{document}
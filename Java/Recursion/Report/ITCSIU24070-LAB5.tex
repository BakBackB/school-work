\documentclass[a4paper,12pt]{article} % Hoặc article nếu là sách ngắn
\usepackage[left=10mm, right=10mm, top=30mm, bottom=30mm]{geometry}
\usepackage[english, vietnamese]{babel}
\usepackage{amssymb}
\usepackage{pdfpages} % Gói để nhúng PDF
\usepackage{hyperref}
\usepackage{babel}
\usepackage{enumitem}
\usepackage{amsmath}
\usepackage{fancyhdr}
\usepackage{bigints}
\usepackage{tikz}
\usepackage{tcolorbox} 
\pagestyle{fancy}
\usepackage{setspace}
\usepackage{hyperref}
\usepackage{listings}
\onehalfspacing % Giãn dòng 1.5 lần
\fancyhead[L]{\textbf{\textit{Algorithms \& Data Structures}}}
\fancyhead[R]{\textbf{\textit{Lab 5}}}
\fancyfoot[C]{\thepage} % Đánh số trang giữa
\setlength{\parindent}{0pt} % Lùi đầu dòng
\setlength{\parskip}{12pt} % Khoảng cách giữa các đoạn
\usepackage{array}
\begin{document}
	\setstretch{1.25}
	\begin{titlepage}
		\begin{center}
			\Large{Trường Đại Học Quốc Tế - ĐHQG TP.HCM}
		\end{center}
		\vspace*{\fill}
		\begin{center}
			{\Huge{\textbf{LAB REPORT}}}\\[0.5cm]
			\date{-5ex}
			\Large{Course: Algorithms \& Data Structures} \large{LAB 5}
		\end{center}
		\vspace*{\fill}
		\large{\textbf{Full Name}: Trần Minh Phúc}\dotfill\par
		\noindent \large{\textbf{Student's ID}: ITCSIU24070}\dotfill
	\end{titlepage}
	\section*{Problem 1: Use the following function puzzle(..) to answer problems 1 - 3.}
	\begin{lstlisting}[
		language=Java,
		showspaces=false,
		basicstyle=\ttfamily,
		numbers=left,
		numberstyle=\tiny,
		commentstyle=\color{gray}
		]
int puzzle(int base, int limit)
{ //base and limit are nonnegative numbers
	if ( base > limit )
	return -1;
	else if ( base == limit )
	return 1;
	else
	return base * puzzle(base + 1, limit);
}
	\end{lstlisting}
	\subsection*{Identify the base case(s) of function puzzle(..)}
	We have already known that the base case define the condition that stops the function from recalling itself again. Therefore, there are two base cases in the given code, which are:
	\begin{lstlisting}[
	language=Java,
	showspaces=false,
	basicstyle=\ttfamily,
	numbers=left,
	numberstyle=\tiny,
	commentstyle=\color{gray}
	]
if ( base > limit )
return -1;
else if ( base == limit )
return 1;
\end{lstlisting}
	\subsection*{Identify the recursive case(s) of function puzzle(..)}
	Since the recursive case recall itself again, we can easily find it in the given code:
	\begin{lstlisting}[
	language=Java,
	showspaces=false,
	basicstyle=\ttfamily,
	numbers=left,
	numberstyle=\tiny,
	commentstyle=\color{gray}
	]
else
	return base * puzzle(base + 1, limit);
	\end{lstlisting}
	\subsection*{What displayed}
	\begin{enumerate}[label = \alph*.]
		\item System.out.println(puzzle(14,10))\par
		Since $base (14) > limit (10)$ - base case, the prinln() method will display $-1$
		\item System.out.println(puzzle(4,7))\par
		Since $base (4) < limit (7)$ - recursive case, the println() method will display $120$, derived from a number of calls $(4*5*6*1)$.
		\item System.out.println(puzzle(0,0))\par
		The println() method would display $1$ as the $base (0) = limit (0)$.
	\end{enumerate}
	\section*{Problem 2: Complete the Java code to recursively evaluate the sum: sum = 1 + 1/2 +
		1/3 +...+1/n, n > 1.}
		\begin{lstlisting}[
		language=Java,
		showspaces=false,
		basicstyle=\ttfamily,
		numbers=left,
		numberstyle=\tiny,
		commentstyle=\color{gray}
		]
double sum(int n)	// n>=1
{
    if(n == 1)
        return 1;
    return 1/n + sum(n-1);
}
	\end{lstlisting}
	\section*{Problem 4: Write a recursive function that finds and returns the minimum element in an array, where the array and its size are given as parameters.}
	\begin{lstlisting}[
		language=Java,
		showspaces=false,
		basicstyle=\ttfamily,
		numbers=left,
		numberstyle=\tiny,
		commentstyle=\color{gray}
		]
public class FindMin {
  int findmin(int[] arr, int n) {
      if (arr == null || arr.length == 0) {
        throw newIllegalArgumentException("Array must not be null or empty");
    }		
      if (n == 1) {
        return arr[0];
      }
    return min(arr[n - 1], findmin(arr, n - 1));
    }
  public int min(int a, int b) {
      if (a > b) {
        return b;
      }
    return a;
  }
}
	\end{lstlisting}
	\section*{Problem 6: Write a method that receives two integers and returns the largest common divisor. The formula to calculate the Largest common divisor is shown below:}
	\begin{center}
		\includegraphics[scale = 1.5]{gcd}
	\end{center}
	\begin{lstlisting}[
	language=Java,
	showspaces=false,
	basicstyle=\ttfamily,
	numbers=left,
	numberstyle=\tiny,
	commentstyle=\color{gray}
	]
public class Gcd {
    public int gcd(int p, int q) {
      if (q == 0) {
        return p;
      }
    return gcd(q, p % q);
    };
}
\end{lstlisting}
	\section*{Problem 8: Write a recursive function to generate all subsets of a given set.}
	\begin{lstlisting}[
		language=Java,
		showspaces=false,
		basicstyle=\ttfamily,
		numbers=left,
		numberstyle=\tiny,
		commentstyle=\color{gray}
		]
import java.util.ArrayList;
import java.util.List;

  public static List<List<Integer>> subsets(int[] nums) {
    List<List<Integer>> result = new ArrayList<>();
    backtrack(result, new ArrayList<>(), nums, 0);
      return result;
    }
	
  private static void backtrack(List<List<Integer>> result, List<Integer> 
  tempList,int[] nums, int start) {
    result.add(new ArrayList<>(tempList));
    for (int i = start; i < nums.length; i++{
      tempList.add(nums[i]);
      backtrack(result, tempList, nums, i + 1);
      tempList.remove(tempList.size() - 1);
    }
  }
}
	\end{lstlisting}
	\section*{Problem 10: Use recursion to generate a Sierpinski triangle fractal}
		\begin{lstlisting}[
		language=Java,
		showspaces=false,
		basicstyle=\ttfamily,
		numbers=left,
		numberstyle=\tiny,
		commentstyle=\color{gray}
		]
public class SierpinskiRecursive {
public void printSierpinski(int n, int y) {
  if (y < 0) {
    return;
  }
  printSpaces(y);
  printLine(0, n, y);  
  System.out.println();
  printSierpinski(n, y - 1);
}

public void printSpaces(int count) {
  if (count == 0) {
    return;
  }
  System.out.print(" ");
  printSpaces(count - 1);
}

public void printLine(int x, int numberOfRows, int y) {
  if (x + y >= numberOfRows) {
  	return;
  }
  if ((x & y) != 0) {
  	System.out.print("  ");
  } else {
  	System.out.print("* ");
  }
  printLine(x + 1, numberOfRows, y);
  }
}	
	\end{lstlisting}

	\begin{center}
		\begin{tikzpicture}
			\draw (-8,0) -- (-3.5,0);
			\node at (0,0) {\textit{This is the end of the report}};
			\draw (3.5,0) -- (8,0);
		\end{tikzpicture}
	\end{center}
\end{document}
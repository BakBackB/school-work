\documentclass[a4paper,12pt]{article} % Hoặc article nếu là sách ngắn
\usepackage[left=10mm, right=10mm, top=30mm, bottom=30mm]{geometry}
\usepackage[english, vietnamese]{babel}
\usepackage{amssymb}
\usepackage{pdfpages} % Gói để nhúng PDF
\usepackage{hyperref}
\usepackage{babel}
\usepackage{amsmath}
\usepackage{fancyhdr}
\usepackage{bigints}
\usepackage{tikz}
\usepackage{tcolorbox} 
\pagestyle{fancy}
\usepackage{setspace}
\usepackage{hyperref}
\onehalfspacing % Giãn dòng 1.5 lần
\fancyhead[L]{\textbf{\textit{Algorithms \& Data Structures}}}
\fancyhead[R]{\textbf{\textit{Lab 7}}}
\fancyfoot[C]{\thepage} % Đánh số trang giữa
\setlength{\parindent}{0pt} % Lùi đầu dòng
\setlength{\parskip}{12pt} % Khoảng cách giữa các đoạn
\usepackage{array}
\begin{document}
	\setstretch{1.25}
	\begin{titlepage}
		\begin{center}
			\Large{Trường Đại Học Quốc Tế - ĐHQG TP.HCM}
		\end{center}
		\vspace*{\fill}
		\begin{center}
			{\Huge{\textbf{LAB REPORT}}}\\[0.5cm]
			\date{-5ex}
			\Large{Course: Algorithms \& Data Structures} \large{LAB 7}
		\end{center}
		\vfill
		\large{\textbf{Full Name}: Trần Minh Phúc}\dotfill\par
		\noindent \large{\textbf{Student's ID}: ITCSIU24070}\dotfill
	\end{titlepage}
	\section{Add a method that counts the elements in a binary tree into the Tree Class. Specifically, the method takes no parameters and returns an integer equal to the number of elements in the tree.}
	\begin{center}
		\includegraphics[width = \textwidth]{1}
	\end{center}
	\section{Add a method that computes the height of a binary tree into the Tree Class. Specifically, this method has no parameters and returns an integer equal to the height of the tree.}
	\begin{center}
		\includegraphics[width = \textwidth]{2}
	\end{center}
	\section{Add a method that counts a binary tree’s leaves tree into the Tree Class. Specifically, this method has no parameters and returns an integer equal to the number of leaves in the tree.}
	\begin{center}
		\includegraphics[width = \textwidth]{3}
	\end{center}
	\section{Add a method that determines whether a binary tree is fully balanced. This method takes no parameters and returns a Boolean value: true if the tree is fully balanced and false if not.}
	\begin{center}
		\includegraphics[width = \textwidth]{4}
	\end{center}
	\section{Define two binary trees to be identical if both are empty or their roots are equal, their left subtrees are identical, and their right subtrees are identical. Design a method that determines whether two binary trees are identical (this method takes a second binary tree as its only parameter and returns a Boolean value: true if the tree receiving the message is identical to the parameter, and false otherwise).}
	\begin{center}
		\includegraphics[width = \textwidth]{5}
	\end{center}
	\section{Huffman coding}
	\begin{center}
		\includegraphics[width = \textwidth]{6}
	\end{center}
	\section{TreeApp.java}
	\begin{itemize}
		\item During the implementation, counters were added to measure the number of comparisons performed by each operation.\par
		
		find(): The number of comparisons grows proportionally to the height of the tree. In a well-balanced BST, this is O(log n), but in an unbalanced tree it can degrade to O(n).\par
		
		insert(): Similar to find(), insertion compares keys while traversing down the tree. Efficiency is also O(h), where h is the height.\par
		
		delete(): Deletion requires locating the node and then adjusting the structure (handling 0, 1, or 2 children). Its comparisons remain O(h), but are typically slightly higher than find() and insert() due to structural adjustments.\par
		\item \begin{center}
			\includegraphics[width = 0.85\textwidth]{7}
		\end{center}
		\item \begin{center}
			\includegraphics[width = 0.85\textwidth]{8}
		\end{center}
		\item \begin{center}
			\includegraphics[width = 0.85\textwidth]{9}
		\end{center}
		\item Different traversals create significantly different BST shapes, proving that traversal sequences do not preserve structural information. Only the set of values is preserved, not the tree’s shape.
	\end{itemize}
	\begin{center}
		\begin{tikzpicture}
			\draw (-8,0) -- (-3.5,0);
			\node at (0,0) {\textit{This is the end of the report}};
			\draw (3.5,0) -- (8,0);
		\end{tikzpicture}
	\end{center}
\end{document}
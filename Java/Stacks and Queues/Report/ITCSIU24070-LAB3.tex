\documentclass[a4paper,12pt]{article} % Hoặc article nếu là sách ngắn
\usepackage[left=10mm, right=10mm, top=30mm, bottom=30mm]{geometry}
\usepackage[english, vietnamese]{babel}
\usepackage{amssymb}
\usepackage{pdfpages} % Gói để nhúng PDF
\usepackage{hyperref}
\usepackage{babel}
\usepackage{amsmath}
\usepackage{fancyhdr}
\usepackage{bigints}
\usepackage{tikz}
\usepackage{tcolorbox} 
\pagestyle{fancy}
\usepackage{setspace}
\usepackage{hyperref}
\onehalfspacing % Giãn dòng 1.5 lần
\fancyhead[L]{\textbf{\textit{Algorithms \& Data Structures}}}
\fancyhead[R]{\textbf{\textit{Lab 3}}}
\fancyfoot[C]{\thepage} % Đánh số trang giữa
\setlength{\parindent}{0pt} % Lùi đầu dòng
\setlength{\parskip}{12pt} % Khoảng cách giữa các đoạn
\usepackage{array}
\begin{document}
	\setstretch{1.25}
	\begin{titlepage}
		\begin{center}
			\Large{Trường Đại Học Quốc Tế - ĐHQG TP.HCM}
		\end{center}
		\vspace*{\fill}
		\begin{center}
			{\Huge{\textbf{LAB REPORT}}}\\[0.5cm]
			\date{-5ex}
			\Large{Course: Algorithms \& Data Structures} \large{LAB 3}
		\end{center}
		\vspace*{\fill}
		\large{\textbf{Full Name}: Trần Minh Phúc}\dotfill\par
		\noindent \large{\textbf{Student's ID}: ITCSIU24070}\dotfill
	\end{titlepage}
	\section{Problem 1}
	\Large{\textbf{Write a program to}}
	\subsection{Convert a decimal number and convert it to octal form}
	\begin{center}
		\includegraphics[scale = 0.3]{dectobin}
	\end{center}
	\subsection{Concatenate two stacks}
	\begin{center}
		\includegraphics[scale = 0.3]{concatenate}
	\end{center}
	\subsection{Determine if the contents of one stack are identical to that of another}
	\begin{center}
		\includegraphics[scale = 0.355]{identical}
	\end{center}
	\section{Problem 2}
	\begin{center}
		\includegraphics[scale = 0.25]{infixtopostfix}\par
		\includegraphics[scale=1.75]{infixtopostfixtoutput}
	\end{center}
	\section{QueueApp.java}
	\subsection{Method to display the queue array and the front and rear indices}
	\begin{center}
		\includegraphics[scale = 0.38]{queuemain}
		\includegraphics[scale = 0.625]{queuearraydisplay}
		\includegraphics[scale = 1.5]{queuedisplayoutput}
	\end{center}
	In this scenario, as the underlying array's index get to the wraparound, it is being reset so that the queue can add item at the rear.
	\subsection{Method to display the queue using loops}
	\begin{center}
		\includegraphics[scale = 0.35]{queuedisplay}
	\end{center}
	\subsection{Simulation}
	\begin{center}
		\includegraphics[scale = 0.325]{simulation}
		\includegraphics[scale = 1.25]{simulationoutput}
	\end{center}
	\section{StackApp.java}
	\subsection{Method to display the stack array and the stack itself}
	\begin{center}
		\includegraphics[scale = 0.5]{stackdisplay}
	\end{center}
	\section{PriorityQApp.java}
	\subsection{Method to display the queue}
	\begin{center}
		\includegraphics[scale = 0.435]{priorityqdisplay}
	\end{center}
	\subsection{Compare queue and priority queue insertion method}
	For inserting at the rear using Priority Queue, it is less efficient than basic Queue since its insertion also sorts the queue.
	\begin{center}
		\includegraphics[scale = 0.35]{priorityqinsert}
	\end{center}
	\subsection{Priority simulation}
	\begin{center}
		\includegraphics[scale = 0.3]{prioritysimulation}
	\end{center}
	\begin{center}
		\begin{tikzpicture}
			\draw (-8,0) -- (-3.5,0);
			\node at (0,0) {\textit{This is the end of the report}};
			\draw (3.5,0) -- (8,0);
		\end{tikzpicture}
	\end{center}
\end{document}